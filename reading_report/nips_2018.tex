\documentclass{article}

% if you need to pass options to natbib, use, e.g.:
% \PassOptionsToPackage{numbers, compress}{natbib}
% before loading nips_2018

% ready for submission
\usepackage[final]{nips_2018}

% to compile a preprint version, e.g., for submission to arXiv, add
% add the [preprint] option:
% \usepackage[preprint]{nips_2018}

% to compile a camera-ready version, add the [final] option, e.g.:
% \usepackage[final]{nips_2018}

% to avoid loading the natbib package, add option nonatbib:
% \usepackage[nonatbib]{nips_2018}

\usepackage[utf8]{inputenc} % allow utf-8 input
\usepackage[T1]{fontenc}    % use 8-bit T1 fonts
\usepackage{hyperref}       % hyperlinks
\usepackage{url}            % simple URL typesetting
\usepackage{graphicx, multicol} % Enhanced support for graphics
\usepackage{xcolor} % Driver-independent color extensions
\usepackage{marvosym, wasysym} % More symbols
\usepackage{rotating} % Rotation tools
\usepackage{censor} % Facilities for controlling restricted text
\usepackage{listings}
%\usepackage{lstlisting} % Environment for non-formatted code, !uses style file!
\usepackage{pseudocode} % Environment for specifying algorithms in a natural way
%\usepackage{style/avm} % Environment for f-structures, !uses style file!
\usepackage{booktabs} % Enhances quality of tables
\usepackage{bm} % Enhances quality of tables
\usepackage{tikz-qtree} % Easy tree drawing tool
\usepackage{booktabs}       % professional-quality tables
\usepackage{amsfonts}       % blackboard math symbols
\usepackage{mathtools} % Mathematical typesetting
\usepackage{nicefrac}       % compact symbols for 1/2, etc.
\usepackage{microtype}      % microtypography
\usepackage{cleveref}
\usepackage{tikz}
\usepackage{placeins}
\usepackage{subcaption}
\usepackage{cancel}
\newcommand*{\vertbar}{\rule[-1ex]{0.5pt}{2.5ex}}
\newcommand*{\horzbar}{\rule[.5ex]{2.5ex}{0.5pt}}

\newcommand{\E}[1]{{\mathbb{E}\left[#1\right]}}
\newcommand{\Et}[2]{{\mathbb{E}_{#2}\left[#1\right]}}
\newcommand{\var}[1]{{\text{var}\left[#1\right]}}
\newcommand{\diffpart}[1]{\frac{\partial}{\partial #1}}
\newcommand{\diffpartt}[2]{\frac{\partial#1}{\partial #2}}
\newcommand{\?}{\stackrel{?}{=}}
\newcommand{\intinf}{\int\limits_{-\infty}^{\infty}}
\newcommand{\intnulinf}{\int\limits_{0}^{\infty}}
\newcommand{\intpi}{\int\limits_{0}^{2\pi}}
\newcommand{\sume}[1]{\sum\limits_{#1}}
\newcommand{\sumt}[2]{\sum\limits_{#1}^{#2}}
\newcommand{\prode}[1]{\prod\limits_{#1}}
\newcommand{\prodt}[2]{\prod\limits_{#1}^{#2}}
\newcommand{\argmin}[1]{\underset{#1}{\mathop{\mathrm{argmin}}}}
\newcommand{\argmax}[1]{\underset{#1}{\mathop{\mathrm{argmax}}}}
\newcommand{\normdist}[1]{\mathcal{N}\left(#1\right)}
\newcommand{\expe}[1]{\exp\left[#1\right]}
\newcommand{\half}{\frac{1}{2}}
\newcommand{\ind}[2]{\mathbb{I}(#1=#2)}

\usepackage{centernot}
\newcommand{\indep}{\mathrel{\text{\scalebox{1.07}{$\perp\mkern-10mu\perp$}}}}
\newcommand{\notindep}{\centernot{\indep}}
\renewcommand{\arraystretch}{1.3}

% VECTOR NOTATION
\newcommand{\mSig}{\bm{\Sigma}}
\newcommand{\mSigi}{\bm{\Sigma}^{-1}}
\newcommand{\mS}{\bm{S}}
\newcommand{\mB}{\bm{B}}
\newcommand{\mA}{\bm{A}}
\newcommand{\mX}{\mathbf{X}}
\newcommand{\mZ}{\mathbf{Z}}
\newcommand{\msX}{\mathcal{X}}
\newcommand{\mM}{\mathbf{M}}
\newcommand{\mV}{\mathbf{V}}
\newcommand{\mY}{\mathbf{Y}}
\newcommand{\mC}{\mathbf{C}}
\newcommand{\mGamma}{\bm{\Gamma}}
\newcommand{\mSi}{\bm{S}^{-1}}
\newcommand{\mI}{\mathbf{I}}
\newcommand{\mL}{\mathbf{L}}
\newcommand{\mLi}{\mathbf{L}^{-1}}
\newcommand{\mPhi}{\bm{\Phi}}
\newcommand{\vnul}{\bm{0}}
\newcommand{\vx}{\mathbf{x}}
\newcommand{\va}{\mathbf{a}}
\newcommand{\vb}{\mathbf{b}}
\newcommand{\vw}{\mathbf{w}}
\newcommand{\vm}{\mathbf{m}}
\newcommand{\vu}{\mathbf{u}}
\newcommand{\vy}{\mathbf{y}}
\newcommand{\vz}{\mathbf{z}}
\newcommand{\vt}{\bm{t}}
\newcommand{\vxi}{\bm{\xi}}
\newcommand{\vmu}{\bm{\mu}}
\newcommand{\valpha}{\bm{\alpha}}
\newcommand{\vbeta}{\bm{\beta}}
\newcommand{\vphi}{\bm{\phi}}
\newcommand{\vpi}{\bm{\pi}}
\newcommand{\vlambda}{\bm{\lambda}}
\newcommand{\veta}{\bm{\eta}}
\newcommand{\Ck}{\mathcal{C}_k}
\newcommand{\mTheta}{\bm{\Theta}}
\newcommand{\mT}{\bm{T}}
\newcommand{\mW}{\bm{W}}
\newcommand{\vtheta}{\bm{\theta}}
\newcommand{\xtil}{\widetilde{x}}

% vim: ft=tex


\title{Reading assignment}

% The \author macro works with any number of authors. There are two
% commands used to separate the names and addresses of multiple
% authors: \And and \AND.
%
% Using \And between authors leaves it to LaTeX to determine where to
% break the lines. Using \AND forces a line break at that point. So,
% if LaTeX puts 3 of 4 authors names on the first line, and the last
% on the second line, try using \AND instead of \And before the third
% author name.

\author{
  Maico Timmerman \\
  Master Artificial Intelligence\\
  University of Amsterdam\\
  \texttt{maico.timmerman@gmail.com} \\
  10542590 \\
  \And
  Daan Smedinga\\
  Master Artificial Intelligence\\
  University of Amsterdam\\
  \texttt{TODO email}\\
  TODO student id
}

\begin{document}
% \nipsfinalcopy is no longer used

\maketitle

% \begin{abstract}
%   The abstract paragraph should be indented \nicefrac{1}{2}~inch
%   (3~picas) on both the left- and right-hand margins. Use 10~point
%   type, with a vertical spacing (leading) of 11~points.  The word
%   \textbf{Abstract} must be centered, bold, and in point size 12. Two
%   line spaces precede the abstract. The abstract must be limited to
%   one paragraph.
% \end{abstract}

\section*{Cloud Computing and Grid Computing 360-Degree Compared}

In this paper an in-depth comparison is made between Grid Computing and Cloud
Computing. Grids are defined by Ian Foster as 1) coordinated resources that are
not subject to centralized control, 2) uses standard, open, general-purpose
protocols and interfaces, and 3) delivers non-trivial qualities of service. The
first two do not clearly hold for Clouds, while the third does. There is little
consensus on how to define Clouds, however the author defines them as: ``A
large-scale distributed computing paradigm that is driven by economies of scale,
in which a pool of abstracted virtualized dynamically-scalable, managed
computing power, storage, platforms, and services are delivered on demand to
external customers over the Internet.''
To further support these definitions, the author compares Grids and Clouds from
different perspectives: architecture, security model, business model,
programming model, virtualization, data model, compute model, provenance and
applications.

Clouds charge the users using creditcards based on per
instance-hours, GB-Month of storage, TB / month data transfer, or similar
metrics. For Grids, mostly found in academia or government labs,  the users
provide proposals for projects, which are assigned service units (like CPU
hours) which they can spend. Architecturally, Clouds can be implemented over
existing Grid technologies. The Grid starting point was to use federated
resources comprising of compute, storage and network resources. These generally
heterogeneous and dynamic resources can come from multiple geographically
distributed institutions: ``Virtual Organizations''. To support these, Grid
protocol architecture define standard protocols, middleware, toolkits, and
services with interoperability and security as their primary concerns. Grids
provide protocols and services at five different layers. 1) The fabric layer are
the resources types such as compute, storage, etc. 2) Connectivity layer defines
communication and authentication. 3) Resource layer defined protocols for
publication, monitoring, accounting and payment. 4) Collective layer capture
interactions acress collections of resources. 5) Application layer comprises the
user application built on all the protocols. For Clouds the protocol
architecture consists of four layers. 1) Fabric layer similar to Grids. 2)
Unified resource layer containing resource that have been abstracted
/encapsulated (usually using virtualization). 3) The platform layer consisting
of a set of specialized tools, middleware, and services to provide development
and/or deployment. 4) Application layer similar to Grids. Clouds generally
provide services at three levels. 1) Infrastructure as a Service (IaaS)
provisioning hardware, software and equipments. 2) Platform as a Service (PaaS)
provisioning high-level integrated environments. 3). Software as a Service
(SaaS) delivers special purpose software remotely accessible for the user.

Resources in Grids are often batch-scheduled, where a resource manager process
requests of users for a number of resources. Based on the user credentials
requests are processed and accounting is done. Clouds resource model is very
different, as all resource in the Cloud are shared by all users at the same
time. This makes providing a good level of QoS a non-trivial task and will
likely be one of the major challenges in Cloud Computing. Data-wise Clouds and
Grids face similar issues when the scale of the data grows every year. Moving
data repeatedly to distant CPUs is becoming a bottleneck. To archive good
scalability Clouds, Grids, and their applications must distribute data over many
computers and computations must be steered towards the best place to execute in
order to minimize communication costs. Critical is the combination of compute
and resources, as data movement will not scale to the peta-scale datasets and
could lead to underutilization of raw resources in both Clouds and Grids. While
Grids often deal with more data and progress has been made in data-aware
schedulers, however Clouds will face the same issues as the scales of the Clouds
grow. Virtualization in Clouds is used to provide necessary abstraction between
fabric and resource overlays. It enables encapsulation, providing better
security, manageability, and isolation in Clouds. Grids do not rely on
virtualization as much as Clouds do, as individual organization maintain full
control of the resources, however there are efforts to use virtualization as
well.

Monitoring in Clouds is a potential difficulty due to the virtualization. As
mentioned, Grid resources are not highly abstract as in clouds, making
monitoring easier. However, it can be argued that monitoring in Clouds is less
important as users are interacting with a more abstract layer that is
potentially more sophisticated. Provence refers to the derivation history of a
data product. It is different from monitoring in the sense that it provides a
overview of the past and supports the discovery and reproducibility of
scientific results. Provence in Clouds seems to be more challenging, as data
needs to be tracked across multiple service providers.

The software written in Grids does not differ fundamentally from traditional
parallel and distributed environments. Grids often target large scale scientific
computations and must scale to leverage large number of resources while
considering reliability and fault tolerance. Clouds seem to consist more of
mesh-up\'s and scripting, since there is no easy way to integrate services and
application from various providers.

In conclusion, the author has created an overview of the Cloud- and Grid
Computing paradigms. While the paper has been written 10 years ago, it gives a
very good overview of the differences in Clouds and Grids still present today.
Especially the concluding words of the author regarding the required protocols
and tools is still relevant today, as currently the tooling for Clouds and Grids
are advancing quickly.


\section*{The Pathologies of Big Data}
The author of this piece is concerned with answering the question: When is data
“big”? The author addresses it indirectly by explaining the limits that are 
encountered as data scales up, and ultimately answers the question in those 
terms. Here follows a paraphrased summary which contains some free 
interpretation:

A dataset containing basic demographic data on every person in the world adds 
up to about 100GB. Answering questions about the whole dataset at once is still
doable on a basic consumer-grade computer within reasonable time, so is it not 
big? Storing the dataset in a PostgreSQL database increased the used disk space
linearly (usually not a problem) and revealed one of the ‘pathologies’ of big 
data: the scaling of algorithms for data analysis. 

Up to the order of thousands, PostgreSQL uses a fast hash table. Beyond that, 
it resorts to sorting by grouping column, which is fine up to the order of 
millions, but it does not scale well up to billions. Transactions concerning a 
handful of items (the typical use case) are still handled with ease in such 
large databases, but a simple query on 1 billion items took more than 24 hours 
to complete! When asking questions about truly large datasets, such as in data 
warehouses, you must think about how to suitably store it to allow interesting 
questions to be answered in reasonable time.

In big data, there are typically many more observations than things that are 
being observed, i.e. the cardinality is low. For example, each phone’s 
location is tracked thousands of times a day. Asking questions about time 
series like this does not scale well in relational databases, as they assume 
that all observations are independent.

One aspect that makes data big is when it severely punishes inefficient access 
patterns. Another one is that it may run into hard limits imposed by software 
(e.g. limited memory address space) or hardware (e.g. limited memory). Single 
computers generally can’t be scaled up more than a magnitude compared to 
consumer grade machines – there’s only so much memory that can fit into one 
computer – so the only practical scaling method is through distributed 
computing. This introduces the network as another bottlenecking layer, 
although networks are generally faster than disk storage in bandwidth and 
latency. Distributed computing only truly scales when a problem can be split 
up into equally sized, independent parts. When the parts are not equally 
sized, the slowest part determines the performance. When parts are not 
independent, communication over networks incur a lot of overhead.

A dataset can then be defined as large enough to cause performance problems 
when no careful thought is put into how to store, access and use it. That means
this definition changes with time as computers and software become faster and 
more intelligent.



Despite being written almost ten years ago, this piece is still just as relevant 
today. The author’s points on scalability of data-driven applications and 
computers still stand, although Optane storage may soon dramatically increase 
capacity of high performance memory capacity. Funnily enough, even his examples 
of ‘current day’ computers from that time don’t seem outdated, as computers 
since then have scaled mostly in terms of parallel processing instead of raw 
speed. It’s hard to say if database software has adapted to deal with 
scalability to billions, but modern solutions have surely improved as big data 
has become a commonplace application as internet usage and availability of large 
datasets has grown enormously in the past ten years.



\end{document}
